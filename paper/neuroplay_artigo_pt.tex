\documentclass[12pt,a4paper]{article}

% Pacotes essenciais
\usepackage[utf8]{inputenc}
\usepackage[T1]{fontenc}
\usepackage[brazilian]{babel}
\usepackage{amsmath,amssymb}
\usepackage{graphicx}
\usepackage{hyperref}
\usepackage{cite}
\usepackage{booktabs}
\usepackage{geometry}
\usepackage{abstract}
\usepackage{authblk}

\geometry{margin=2.5cm}

% Configurações do hyperref
\hypersetup{
    colorlinks=true,
    linkcolor=blue,
    citecolor=blue,
    urlcolor=blue
}

\title{\textbf{NeuroPlay: Uma Plataforma Web Adaptativa para Treinamento de Funções Executivas em Indivíduos com Transtorno do Espectro Autista}}

\author[1]{Equipe de Pesquisa}
\affil[1]{Departamento de Ciência da Computação e Neurociência Cognitiva}

\date{\today}

\begin{document}

\maketitle

\begin{abstract}
\textbf{Contexto:} O Transtorno do Espectro Autista (TEA) é caracterizado por déficits em funções executivas (FE), incluindo memória de trabalho, controle inibitório e flexibilidade cognitiva. Intervenções digitais têm demonstrado potencial para abordar esses desafios, com gamificação e aprendizado adaptativo emergindo como estratégias eficazes.

\textbf{Objetivo:} Este artigo apresenta o NeuroPlay, uma plataforma web projetada para treinar funções executivas em indivíduos com TEA através de jogos terapêuticos gamificados e adaptativos. A plataforma integra inteligência artificial para ajuste personalizado de dificuldade, princípios de design informados pela neurodiversidade e paradigmas de treinamento cognitivo baseados em evidências.

\textbf{Métodos:} O NeuroPlay foi desenvolvido utilizando React.js para o frontend, Three.js para visualização 3D e TensorFlow.js para aprendizado de máquina no cliente. A plataforma implementa três jogos principais direcionados a domínios específicos de FE: (1) Dual N-Back para memória de trabalho, (2) Caçador de Alvos para controle inibitório, e (3) Mestre do Sinal para flexibilidade cognitiva. O design seguiu os padrões de acessibilidade WCAG 2.1 AA e diretrizes específicas para neurodiversidade.

\textbf{Resultados:} A arquitetura da plataforma integra com sucesso algoritmos adaptativos de IA que ajustam a dificuldade das tarefas em tempo real com base no desempenho do usuário. O sistema alcança compatibilidade multiplataforma, opções de customização sensorial e rastreamento abrangente de progresso para educadores e cuidadores.

\textbf{Conclusões:} O NeuroPlay representa uma solução escalável e acessível para treinamento de FE em populações com TEA. Trabalhos futuros incluirão estudos de validação clínica, expansão de módulos de cognição social e integração de feedback biométrico multimodal.

\textbf{Palavras-chave:} Transtorno do Espectro Autista, Funções Executivas, Gamificação, Aprendizado Adaptativo, Intervenção Baseada na Web, Neurodiversidade, Acessibilidade
\end{abstract}


\section{Introdução}

O Transtorno do Espectro Autista (TEA) é uma condição do neurodesenvolvimento que afeta aproximadamente 1 em cada 100 crianças globalmente \cite{WHO2023}. Além das características diagnósticas centrais de diferenças na comunicação social e comportamentos restritos/repetitivos, indivíduos com TEA frequentemente exibem déficits em funções executivas (FE) que impactam significativamente o funcionamento diário, desempenho acadêmico e qualidade de vida \cite{Demetriou2018}.

\subsection{Funções Executivas no Autismo}

As funções executivas englobam um conjunto de processos cognitivos de ordem superior, incluindo memória de trabalho (MT), controle inibitório (CI) e flexibilidade cognitiva (FC) \cite{Diamond2013}. Pesquisas demonstram consistentemente que indivíduos com TEA apresentam prejuízos nesses domínios:

\begin{itemize}
    \item \textbf{Memória de Trabalho:} Déficits tanto em MT verbal quanto espacial são bem documentados em populações com TEA, afetando a capacidade de manter e manipular informações temporariamente \cite{Barendse2013}.
    \item \textbf{Controle Inibitório:} Dificuldades em suprimir respostas prepotentes contribuem para rigidez comportamental e perseveração \cite{Geurts2014}.
    \item \textbf{Flexibilidade Cognitiva:} Habilidades prejudicadas de mudança de conjunto manifestam-se como dificuldade em adaptar-se a regras ou contextos em mudança, contribuindo para a necessidade característica de mesmice no TEA \cite{Leung2016}.
\end{itemize}

Meta-análises recentes indicam que déficits de FE no TEA não são meramente secundários a outros sintomas, mas representam uma característica central da condição \cite{Lai2017}. Importante, esses déficits são passíveis de intervenção, com tamanhos de efeito variando de médio a grande (d de Cohen = 0,45-0,59) para programas de treinamento direcionados \cite{Frontiers2025}.

\subsection{Intervenções Digitais para Autismo}

A última década testemunhou uma proliferação de intervenções baseadas em tecnologia para TEA, impulsionadas por várias vantagens sobre abordagens tradicionais:

\begin{enumerate}
    \item \textbf{Escalabilidade:} Plataformas digitais podem alcançar populações carentes com acesso limitado a serviços especializados.
    \item \textbf{Consistência:} Tarefas computadorizadas fornecem protocolos de treinamento padronizados e replicáveis.
    \item \textbf{Engajamento:} Elementos de gamificação aumentam a motivação e adesão, particularmente para crianças e adolescentes.
    \item \textbf{Coleta de Dados:} Rastreamento automatizado de desempenho permite monitoramento preciso de progresso e resultados.
\end{enumerate}

Uma revisão sistemática de 2025 com 1.801 pacientes com TEA descobriu que jogos terapêuticos digitais produziram melhorias significativas em habilidades sociais, capacidades cognitivas e resultados comportamentais \cite{Frontiers2025Pediatrics}. Notavelmente, sistemas adaptativos que ajustam a dificuldade em tempo real com base no desempenho do usuário demonstraram eficácia superior em comparação com intervenções estáticas \cite{Restack2024}.

\subsection{Neurodiversidade e Design Acessível}

O paradigma da neurodiversidade reformula o autismo como uma variação natural na cognição humana, em vez de um déficit a ser corrigido \cite{Singer1998}. Esta perspectiva tem implicações profundas para o design de intervenções, enfatizando:

\begin{itemize}
    \item Respeito por diferenças individuais no processamento sensorial
    \item Interfaces customizáveis que acomodam necessidades diversas
    \item Abordagens baseadas em forças que aproveitam as habilidades únicas de indivíduos autistas
    \item Design participativo envolvendo stakeholders autistas
\end{itemize}

As Diretrizes de Acessibilidade para Conteúdo Web (WCAG) 2.1 fornecem uma base para design acessível, mas considerações específicas para neurodiversidade se estendem além da conformidade padrão \cite{WCAG2023}. Princípios-chave incluem minimizar sobrecarga sensorial, fornecer estrutura e previsibilidade claras, oferecer múltiplos meios de representação e apoiar a autorregulação \cite{Devqube2025}.

\subsection{Objetivos}

Este artigo apresenta o NeuroPlay, uma plataforma web projetada para abordar as lacunas identificadas nas intervenções atuais de treinamento de FE para TEA. Objetivos específicos incluem:

\begin{enumerate}
    \item Desenvolver uma plataforma acessível e gamificada direcionada aos domínios centrais de FE (MT, CI, FC)
    \item Implementar algoritmos adaptativos de IA para ajuste personalizado de dificuldade
    \item Integrar princípios de design informados pela neurodiversidade em toda a experiência do usuário
    \item Criar um painel abrangente para educadores/cuidadores para monitoramento de progresso
    \item Garantir compatibilidade multiplataforma e funcionalidade offline
\end{enumerate}

