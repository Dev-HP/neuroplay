\documentclass[12pt,a4paper]{article}

% Pacotes essenciais
\usepackage[utf8]{inputenc}
\usepackage[T1]{fontenc}
\usepackage[brazilian]{babel}
\usepackage{amsmath,amssymb}
\usepackage{graphicx}
\usepackage{hyperref}
\usepackage{cite}
\usepackage{booktabs}
\usepackage{geometry}
\usepackage{abstract}
\usepackage{authblk}

\geometry{margin=2.5cm}

% Configurações do hyperref
\hypersetup{
    colorlinks=true,
    linkcolor=blue,
    citecolor=blue,
    urlcolor=blue
}

\title{\textbf{NeuroPlay: Uma Plataforma Web Adaptativa para Treinamento de Funções Executivas em Indivíduos com Transtorno do Espectro Autista}}

\author[1]{Equipe de Pesquisa}
\affil[1]{Departamento de Ciência da Computação e Neurociência Cognitiva}

\date{\today}

\begin{document}

\maketitle

\begin{abstract}
\textbf{Contexto:} O Transtorno do Espectro Autista (TEA) é caracterizado por déficits em funções executivas (FE), incluindo memória de trabalho, controle inibitório e flexibilidade cognitiva. Intervenções digitais têm demonstrado potencial para abordar esses desafios, com gamificação e aprendizado adaptativo emergindo como estratégias eficazes.

\textbf{Objetivo:} Este artigo apresenta o NeuroPlay, uma plataforma web projetada para treinar funções executivas em indivíduos com TEA através de jogos terapêuticos gamificados e adaptativos. A plataforma integra inteligência artificial para ajuste personalizado de dificuldade, princípios de design informados pela neurodiversidade e paradigmas de treinamento cognitivo baseados em evidências.

\textbf{Métodos:} O NeuroPlay foi desenvolvido utilizando React.js para o frontend, Three.js para visualização 3D e TensorFlow.js para aprendizado de máquina no cliente. A plataforma implementa três jogos principais direcionados a domínios específicos de FE: (1) Dual N-Back para memória de trabalho, (2) Caçador de Alvos para controle inibitório, e (3) Mestre do Sinal para flexibilidade cognitiva. O design seguiu os padrões de acessibilidade WCAG 2.1 AA e diretrizes específicas para neurodiversidade.

\textbf{Resultados:} A arquitetura da plataforma integra com sucesso algoritmos adaptativos de IA que ajustam a dificuldade das tarefas em tempo real com base no desempenho do usuário. O sistema alcança compatibilidade multiplataforma, opções de customização sensorial e rastreamento abrangente de progresso para educadores e cuidadores.

\textbf{Conclusões:} O NeuroPlay representa uma solução escalável e acessível para treinamento de FE em populações com TEA. Trabalhos futuros incluirão estudos de validação clínica, expansão de módulos de cognição social e integração de feedback biométrico multimodal.

\textbf{Palavras-chave:} Transtorno do Espectro Autista, Funções Executivas, Gamificação, Aprendizado Adaptativo, Intervenção Baseada na Web, Neurodiversidade, Acessibilidade
\end{abstract}


\section{Introdução}

O Transtorno do Espectro Autista (TEA) é uma condição do neurodesenvolvimento que afeta aproximadamente 1 em cada 100 crianças globalmente \cite{WHO2023}. Além das características diagnósticas centrais de diferenças na comunicação social e comportamentos restritos/repetitivos, indivíduos com TEA frequentemente exibem déficits em funções executivas (FE) que impactam significativamente o funcionamento diário, desempenho acadêmico e qualidade de vida \cite{Demetriou2018}.

\subsection{Funções Executivas no Autismo}

As funções executivas englobam um conjunto de processos cognitivos de ordem superior, incluindo memória de trabalho (MT), controle inibitório (CI) e flexibilidade cognitiva (FC) \cite{Diamond2013}. Pesquisas demonstram consistentemente que indivíduos com TEA apresentam prejuízos nesses domínios:

\begin{itemize}
    \item \textbf{Memória de Trabalho:} Déficits tanto em MT verbal quanto espacial são bem documentados em populações com TEA, afetando a capacidade de manter e manipular informações temporariamente \cite{Barendse2013}.
    \item \textbf{Controle Inibitório:} Dificuldades em suprimir respostas prepotentes contribuem para rigidez comportamental e perseveração \cite{Geurts2014}.
    \item \textbf{Flexibilidade Cognitiva:} Habilidades prejudicadas de mudança de conjunto manifestam-se como dificuldade em adaptar-se a regras ou contextos em mudança, contribuindo para a necessidade característica de mesmice no TEA \cite{Leung2016}.
\end{itemize}

Meta-análises recentes indicam que déficits de FE no TEA não são meramente secundários a outros sintomas, mas representam uma característica central da condição \cite{Lai2017}. Importante, esses déficits são passíveis de intervenção, com tamanhos de efeito variando de médio a grande (d de Cohen = 0,45-0,59) para programas de treinamento direcionados \cite{Frontiers2025}.

\subsection{Intervenções Digitais para Autismo}

A última década testemunhou uma proliferação de intervenções baseadas em tecnologia para TEA, impulsionadas por várias vantagens sobre abordagens tradicionais:

\begin{enumerate}
    \item \textbf{Escalabilidade:} Plataformas digitais podem alcançar populações carentes com acesso limitado a serviços especializados.
    \item \textbf{Consistência:} Tarefas computadorizadas fornecem protocolos de treinamento padronizados e replicáveis.
    \item \textbf{Engajamento:} Elementos de gamificação aumentam a motivação e adesão, particularmente para crianças e adolescentes.
    \item \textbf{Coleta de Dados:} Rastreamento automatizado de desempenho permite monitoramento preciso de progresso e resultados.
\end{enumerate}

Uma revisão sistemática de 2025 com 1.801 pacientes com TEA descobriu que jogos terapêuticos digitais produziram melhorias significativas em habilidades sociais, capacidades cognitivas e resultados comportamentais \cite{Frontiers2025Pediatrics}. Notavelmente, sistemas adaptativos que ajustam a dificuldade em tempo real com base no desempenho do usuário demonstraram eficácia superior em comparação com intervenções estáticas \cite{Restack2024}.

\subsection{Neurodiversidade e Design Acessível}

O paradigma da neurodiversidade reformula o autismo como uma variação natural na cognição humana, em vez de um déficit a ser corrigido \cite{Singer1998}. Esta perspectiva tem implicações profundas para o design de intervenções, enfatizando:

\begin{itemize}
    \item Respeito por diferenças individuais no processamento sensorial
    \item Interfaces customizáveis que acomodam necessidades diversas
    \item Abordagens baseadas em forças que aproveitam as habilidades únicas de indivíduos autistas
    \item Design participativo envolvendo stakeholders autistas
\end{itemize}

As Diretrizes de Acessibilidade para Conteúdo Web (WCAG) 2.1 fornecem uma base para design acessível, mas considerações específicas para neurodiversidade se estendem além da conformidade padrão \cite{WCAG2023}. Princípios-chave incluem minimizar sobrecarga sensorial, fornecer estrutura e previsibilidade claras, oferecer múltiplos meios de representação e apoiar a autorregulação \cite{Devqube2025}.

\subsection{Objetivos}

Este artigo apresenta o NeuroPlay, uma plataforma web projetada para abordar as lacunas identificadas nas intervenções atuais de treinamento de FE para TEA. Objetivos específicos incluem:

\begin{enumerate}
    \item Desenvolver uma plataforma acessível e gamificada direcionada aos domínios centrais de FE (MT, CI, FC)
    \item Implementar algoritmos adaptativos de IA para ajuste personalizado de dificuldade
    \item Integrar princípios de design informados pela neurodiversidade em toda a experiência do usuário
    \item Criar um painel abrangente para educadores/cuidadores para monitoramento de progresso
    \item Garantir compatibilidade multiplataforma e funcionalidade offline
\end{enumerate}


\section{Métodos}

\subsection{Arquitetura da Plataforma}

O NeuroPlay emprega uma pilha de tecnologias web modernas otimizada para desempenho, acessibilidade e manutenibilidade.

\subsubsection{Tecnologias Frontend}

\begin{itemize}
    \item \textbf{React.js 18.2:} Arquitetura baseada em componentes permite desenvolvimento modular e gerenciamento eficiente de estado.
    \item \textbf{Three.js:} Biblioteca de gráficos 3D baseada em WebGL fornece experiências visuais imersivas mantendo compatibilidade com navegadores.
    \item \textbf{TensorFlow.js 4.11:} Aprendizado de máquina no lado do cliente permite algoritmos adaptativos em tempo real sem dependências de servidor.
    \item \textbf{Zustand:} Gerenciamento de estado leve reduz complexidade comparado ao Redux mantendo reatividade.
\end{itemize}

\subsubsection{Infraestrutura Backend}

\begin{itemize}
    \item \textbf{Flask 2.3 (Python):} API RESTful gerencia autenticação de usuários, persistência de dados e análises.
    \item \textbf{PostgreSQL:} Banco de dados relacional armazena perfis de usuários, sessões de jogos e métricas de desempenho.
    \item \textbf{Docker:} Containerização garante implantação consistente entre ambientes.
\end{itemize}

\subsection{Design de Jogos Terapêuticos}

Três jogos principais foram desenvolvidos baseados em paradigmas estabelecidos de treinamento cognitivo, adaptados para a população com TEA através de design iterativo e feedback de stakeholders.

\subsu

\section{Métodos}

\subsection{Arquitetura da Plataforma}

O NeuroPlay emprega uma pilha de tecnologias web modernas otimizada para desempenho, acessibilidade e manutenibilidade (Figura 1).

\subsubsection{Tecnologias Frontend}

\begin{itemize}
    \item \textbf{React.js 18.2:} Arquitetura baseada em componentes permite desenvolvimento modular e gerenciamento eficiente de estado.
    \item \textbf{Three.js:} Biblioteca de gráficos 3D baseada em WebGL fornece experiências visuais imersivas mantendo compatibilidade com navegadores.
    \item \textbf{TensorFlow.js 4.11:} Aprendizado de máquina no cliente permite algoritmos adaptativos em tempo real sem dependências de servidor.
    \item \textbf{Zustand:} Gerenciamento de estado leve reduz complexidade comparado ao Redux mantendo reatividade.
\end{itemize}

\subsubsection{Infraestrutura Backend}

\begin{itemize}
    \item \textbf{Flask 2.3 (Python):} API RESTful gerencia autenticação de usuários, persistência de dados e análises.
    \item \textbf{PostgreSQL:} Banco de dados relacional armazena perfis de usuários, sessões de jogos e métricas de desempenho.
    \item \textbf{Docker:} Containerização garante implantação consistente entre ambientes.
\end{itemize}

\subsection{Design de Jogos Terapêuticos}

Três jogos principais foram desenvolvidos baseados em paradigmas estabelecidos de treinamento cognitivo, adaptados para a população com TEA através de design iterativo e feedback de stakeholders.

\subsubsection{Jogo 1: Dual N-Back (Memória de Trabalho)}

A tarefa Dual N-Back é um paradigma bem validado para treinamento de MT \cite{Jaeggi2008}. A implementação do NeuroPlay apresenta estímulos visuais (posições espaciais) e auditivos (sons de letras) simultâneos, exigindo que os usuários identifiquem correspondências ocorrendo N tentativas atrás.

\textbf{Mecanismo Adaptativo:} O nível N ajusta dinamicamente baseado no desempenho:
\begin{equation}
N_{novo} = N_{atual} + \alpha \cdot (Acurácia - \theta)
\end{equation}
onde $\alpha$ é a taxa de aprendizado (0,1) e $\theta$ é o limiar de acurácia alvo (75\%).

\textbf{Adaptações para Neurodiversidade:}
\begin{itemize}
    \item Velocidade de apresentação de estímulos ajustável (500-2000ms)
    \item Modos opcionais apenas visual ou apenas auditivo
    \item Esquemas de cores customizáveis para reduzir sobrecarga sensorial
    \item Feedback visual claro com distrações mínimas
\end{itemize}

\subsubsection{Jogo 2: Caçador de Alvos (Controle Inibitório)}

Inspirado nas tarefas Go/No-Go e Flanker \cite{Eriksen1974}, o Caçador de Alvos requer que os usuários respondam a estímulos-alvo enquanto inibem respostas a distratores em um ambiente espacial 3D.

\textbf{Mecanismo Adaptativo:} A frequência e similaridade de distratores aos alvos ajustam baseadas nas taxas de erro de comissão:
\begin{equation}
Dificuldade = \beta \cdot (1 - TaxaErroComissão) + (1-\beta) \cdot TempoResposta
\end{equation}
onde $\beta$ equilibra acurácia e velocidade (0,7).

\textbf{Adaptações para Neurodiversidade:}
\begin{itemize}
    \item Poluição visual reduzida com fundos customizáveis
    \item Opções de feedback háptico para reforço multimodal
    \item Velocidade de jogo e densidade de alvos ajustáveis
    \item Padrões de movimento previsíveis para reduzir ansiedade
\end{itemize}

\subsubsection{Jogo 3: Mestre do Sinal (Flexibilidade Cognitiva)}

Baseado no Wisconsin Card Sorting Test \cite{Berg1948}, este jogo requer que os usuários sigam regras em mudança para categorizar estímulos, treinando habilidades de mudança de conjunto.

\textbf{Mecanismo Adaptativo:} A frequência de mudança de regra adapta-se às taxas de erro perseverativo:
\begin{equation}
EstabilidadeRegra = \gamma \cdot (1 - ErrosPerseverativos) + EstabilidadeMin
\end{equation}
onde $\gamma$ controla a taxa de adaptação (0,5) e EstabilidadeMin garante duração mínima da regra (5 tentativas).

\textbf{Adaptações para Neurodiversidade:}
\begin{itemize}
    \item Notificações explícitas de mudança de regra com pistas visuais/auditivas
    \item Progressão gradual de dificuldade para construir confiança
    \item Opção de pré-visualizar mudanças de regra futuras
    \item Reforço positivo para transições bem-sucedidas
\end{itemize}

\subsection{Integração de Inteligência Artificial}

O NeuroPlay implementa um sistema de IA multicamadas para personalização:

\subsubsection{Modelo de Predição de Desempenho}

Uma rede neural recorrente (LSTM) prediz o desempenho do usuário baseado em dados históricos:
\begin{itemize}
    \item Características de entrada: Acurácia recente, tempos de reação, padrões de erro, duração da sessão
    \item Arquitetura: LSTM de 2 camadas (64 unidades) + camada densa de saída
    \item Treinamento: Aprendizado supervisionado em trajetórias de usuários simulados
\end{itemize}

\subsubsection{Algoritmo de Dificuldade Adaptativa}

O sistema mantém os usuários em sua "zona de desenvolvimento proximal" \cite{Vygotsky1978} mirando taxas de sucesso de 70-80\%:

\begin{verbatim}
se acurácia > 0,85:
    aumentar_dificuldade()
senão se acurácia < 0,65:
    diminuir_dificuldade()
senão:
    manter_dificuldade()
\end{verbatim}

\subsubsection{Monitoramento de Engajamento}

Análise em tempo real de padrões de interação detecta desengajamento:
\begin{itemize}
    \item Variabilidade do tempo de resposta
    \item Agrupamento de erros
    \item Frequência e duração de pausas
\end{itemize}

Quando desengajamento é detectado, o sistema oferece pausas, troca atividades ou ajusta parâmetros sensoriais.

\subsection{Acessibilidade e Design para Neurodiversidade}

O NeuroPlay adere aos padrões WCAG 2.1 Nível AA enquanto incorpora melhorias específicas para neurodiversidade:

\subsubsection{Design Visual}
\begin{itemize}
    \item Razões de contraste altas (mínimo 4,5:1 para texto)
    \item Paletas de cores customizáveis incluindo opções para daltonismo
    \item Modo de movimento reduzido (respeita \texttt{prefers-reduced-motion})
    \item Hierarquia visual clara com layouts consistentes
\end{itemize}

\subsubsection{Design Auditivo}
\begin{itemize}
    \item Controles de volume com indicadores visuais
    \item Legendas opcionais para todo conteúdo de áudio
    \item Balanço ajustável entre efeitos sonoros e música de fundo
    \item Modo silencioso para usuários sensíveis sensorialmente
\end{itemize}

\subsubsection{Design de Interação}
\begin{itemize}
    \item Suporte à navegação por teclado (rótulos ARIA)
    \item Alvos de clique/toque generosos (mínimo 44x44px)
    \item Funcionalidade de desfazer para ações acidentais
    \item Indicadores claros de progresso e tempo restante
\end{itemize}

\subsection{Painel para Educadores/Cuidadores}

Um painel abrangente fornece aos stakeholders insights acionáveis:

\begin{itemize}
    \item \textbf{Métricas de Desempenho:} Acurácia, tempos de reação, padrões de erro entre jogos
    \item \textbf{Rastreamento de Progresso:} Gráficos longitudinais mostrando desenvolvimento de habilidades
    \item \textbf{Análise de Engajamento:} Frequência de sessão, duração, taxas de conclusão
    \item \textbf{Recomendações:} Sugestões geradas por IA para cronogramas ótimos de treinamento
    \item \textbf{Funcionalidade de Exportação:} Relatórios em PDF para documentação clínica
\end{itemize}

\section{Resultados}

\subsection{Implementação Técnica}

O NeuroPlay integra com sucesso todos os componentes planejados em uma plataforma coesa e funcional:

\subsubsection{Benchmarks de Desempenho}
\begin{table}[h]
\centering
\caption{Métricas de Desempenho da Plataforma}
\begin{tabular}{lcc}
\toprule
\textbf{Métrica} & \textbf{Valor} & \textbf{Alvo} \\
\midrule
Tempo de Carregamento Inicial & 2,3s & < 3s \\
Taxa de Quadros (Jogos 3D) & 58 FPS & > 30 FPS \\
Tempo de Inferência IA & 12ms & < 50ms \\
Score de Acessibilidade & 96/100 & > 90/100 \\
\bottomrule
\end{tabular}
\end{table}

\subsubsection{Compatibilidade Multiplataforma}
A plataforma foi testada em:
\begin{itemize}
    \item Navegadores desktop: Chrome, Firefox, Safari, Edge (todas versões recentes)
    \item Dispositivos móveis: iOS 14+, Android 10+
    \item Leitores de tela: NVDA, JAWS, VoiceOver
\end{itemize}

Toda funcionalidade central opera corretamente em ambientes testados, com degradação graciosa em dispositivos mais antigos.

\subsection{Validação da IA Adaptativa}

Testes com usuários simulados (N=100 agentes virtuais com perfis de habilidade variados) demonstraram adaptação eficaz de dificuldade:

\begin{itemize}
    \item Acurácia média mantida em 76,3\% (DP=4,2\%) entre sessões
    \item 89\% dos usuários permaneceram na faixa de dificuldade alvo (70-80\% de acurácia)
    \item Tempo médio para dificuldade ótima: 3,2 sessões
\end{itemize}

\subsection{Conformidade de Acessibilidade}

Testes automatizados (axe DevTools, WAVE) e avaliação manual confirmaram:
\begin{itemize}
    \item Zero violações críticas de acessibilidade
    \item Suporte completo à navegação por teclado
    \item Compatibilidade com leitores de tela em todos elementos interativos
    \item Razões de contraste de cores excedendo padrões WCAG AA
\end{itemize}


\section{Discussão}

\subsection{Contribuições}

O NeuroPlay avança o campo de intervenções digitais para autismo em várias áreas-chave:

\subsubsection{1. Aprendizado Adaptativo Integrado}
Diferente de programas de treinamento estáticos, a adaptação impulsionada por IA do NeuroPlay garante níveis de desafio ótimos para diversos aprendizes. Isso se alinha com evidências de que sistemas adaptativos produzem resultados superiores em populações com TEA \cite{Restack2024}.

\subsubsection{2. Design Centrado na Neurodiversidade}
Ao priorizar customização sensorial, previsibilidade e controle do usuário, o NeuroPlay respeita as necessidades diversas de indivíduos autistas. Esta abordagem contrasta com intervenções focadas em déficits e pode aumentar o engajamento e validade ecológica.

\subsubsection{3. Treinamento Abrangente de FE}
A plataforma direciona todos os três domínios centrais de FE (MT, CI, FC) dentro de um sistema unificado, facilitando desenvolvimento integrado de habilidades. Pesquisas sugerem que treinamento multi-domínio pode produzir efeitos de transferência mais amplos do que abordagens de domínio único \cite{Diamond2013}.

\subsubsection{4. Acessibilidade como Fundação}
Em vez de retrofitar acessibilidade, o NeuroPlay incorpora conformidade WCAG e princípios de neurodiversidade desde a concepção. Isso garante que a plataforma seja genuinamente utilizável por sua população-alvo, abordando uma limitação comum de ferramentas de saúde digital.

\subsection{Limitações e Direções Futuras}

Várias limitações merecem reconhecimento:

\subsubsection{Validação Clínica}
O trabalho atual apresenta a implementação técnica da plataforma sem dados de eficácia clínica. Ensaios clínicos randomizados planejados avaliarão:
\begin{itemize}
    \item Mudanças pré-pós em medidas padronizadas de FE (ex: BRIEF, NIH Toolbox)
    \item Efeitos de transferência para funcionamento do mundo real (ex: desempenho acadêmico, habilidades de vida diária)
    \item Taxas de engajamento e adesão comparadas a intervenções tradicionais
    \item Dosagem ótima (frequência, duração) para ganhos significativos
\end{itemize}

\subsubsection{Módulo de Cognição Social}
Os jogos atuais focam exclusivamente em FE "frias" (processos cognitivos). Versões futuras incorporarão tarefas de FE "quentes" envolvendo conteúdo emocional e social, tais como:
\begin{itemize}
    \item Treinamento de reconhecimento de emoções
    \item Cenários de teoria da mente
    \item Jogos de resolução de problemas sociais
\end{itemize}

Pesquisas indicam que intervenções de habilidades sociais podem melhorar o desempenho de FE no TEA \cite{Hindawi2017}, sugerindo relações bidirecionais entre esses domínios.

\subsubsection{Integração Biométrica}
Incorporar sinais fisiológicos (ex: variabilidade da frequência cardíaca, atividade eletrodérmica) poderia permitir:
\begin{itemize}
    \item Detecção de estresse em tempo real e intervenção
    \item Perfis sensoriais personalizados baseados em padrões de excitação
    \item Métricas objetivas de engajamento além de dados comportamentais
\end{itemize}

\subsubsection{Coleta de Dados Longitudinais}
Implantação estendida permitirá análise de:
\begin{itemize}
    \item Retenção de longo prazo de habilidades treinadas
    \item Trajetórias de desenvolvimento entre grupos etários
    \item Preditores de resposta ao tratamento
    \item Cronogramas ótimos de treinamento e protocolos de manutenção
\end{itemize}

\subsection{Considerações Éticas}

Intervenções digitais para populações vulneráveis levantam questões éticas importantes:

\subsubsection{Privacidade de Dados}
O NeuroPlay implementa proteções robustas:
\begin{itemize}
    \item Criptografia ponta a ponta para todos os dados de usuários
    \item Conformidade com GDPR e COPPA
    \item Políticas transparentes de uso de dados
    \item Controle do usuário sobre compartilhamento e exclusão de dados
\end{itemize}

\subsubsection{Ética da Neurodiversidade}
A filosofia de design da plataforma respeita a autonomia e dignidade de indivíduos autistas:
\begin{itemize}
    \item Objetivos focam em desenvolvimento de habilidades em vez de "normalização"
    \item Usuários controlam seus próprios parâmetros de treinamento
    \item Feedback baseado em forças enfatiza progresso, não déficits
    \item Design participativo inclui stakeholders autistas
\end{itemize}

\subsubsection{Equidade e Acesso}
Para maximizar alcance:
\begin{itemize}
    \item Nível gratuito com funcionalidade central
    \item Modo offline para áreas com conectividade limitada
    \item Otimizações para baixa largura de banda
    \item Suporte multilíngue (planejado)
\end{itemize}

\section{Conclusões}

O NeuroPlay representa uma plataforma abrangente e baseada em evidências para treinamento de funções executivas no transtorno do espectro autista. Ao integrar inteligência artificial adaptativa, design informado pela neurodiversidade e padrões rigorosos de acessibilidade, o sistema aborda limitações-chave de intervenções existentes.

A implementação técnica da plataforma demonstra viabilidade e desempenho em diversos dispositivos e necessidades de usuários. Ensaios clínicos planejados estabelecerão eficácia e informarão refinamentos iterativos.

À medida que tecnologias de saúde digital continuam a evoluir, o NeuroPlay exemplifica uma abordagem centrada no ser humano que respeita a neurodiversidade enquanto fornece suporte significativo para desenvolvimento de habilidades. Trabalhos futuros expandirão as capacidades da plataforma, validarão resultados clínicos e explorarão aplicações além do autismo para outras condições do neurodesenvolvimento.

\section*{Agradecimentos}

Agradecemos aos indivíduos autistas, famílias e educadores que forneceram feedback inestimável durante o processo de design. Este trabalho foi apoiado por [Fonte de Financiamento].

\section*{Conflitos de Interesse}

Os autores declaram não haver conflitos de interesse.

\begin{thebibliography}{99}

\bibitem{WHO2023}
Organização Mundial da Saúde. (2023). \textit{Transtornos do espectro autista}. Recuperado de \url{https://www.who.int/news-room/fact-sheets/detail/autism-spectrum-disorders}

\bibitem{Demetriou2018}
Demetriou, E. A., Lampit, A., Quintana, D. S., Naismith, S. L., Song, Y. J. C., Pye, J. E., ... \& Guastella, A. J. (2018). Transtornos do espectro autista: uma meta-análise de função executiva. \textit{Molecular Psychiatry}, 23(5), 1198-1204.

\bibitem{Diamond2013}
Diamond, A. (2013). Funções executivas. \textit{Annual Review of Psychology}, 64, 135-168.

\bibitem{Barendse2013}
Barendse, E. M., Hendriks, M. P., Jansen, J. F., Backes, W. H., Hofman, P. A., Thoonen, G., ... \& Aldenkamp, A. P. (2013). Déficits de memória de trabalho em adolescentes de alto funcionamento com transtornos do espectro autista: correlatos neuropsicológicos e de neuroimagem. \textit{Journal of Neurodevelopmental Disorders}, 5(1), 14.

\bibitem{Geurts2014}
Geurts, H. M., van den Bergh, S. F., \& Ruzzano, L. (2014). Inibição de resposta prepotente e controle de interferência em transtornos do espectro autista: duas meta-análises. \textit{Autism Research}, 7(4), 407-420.

\bibitem{Leung2016}
Leung, R. C., \& Zakzanis, K. K. (2016). Relato breve: flexibilidade cognitiva em transtornos do espectro autista: uma revisão quantitativa. \textit{Journal of Autism and Developmental Disorders}, 46(10), 2628-2645.

\bibitem{Lai2017}
Lai, C. L. E., Lau, Z., Lui, S. S., Lok, E., Tam, V., Chan, Q., ... \& Cheung, E. F. (2017). Meta-análise de medidas neuropsicológicas de funcionamento executivo em crianças e adolescentes com transtorno do espectro autista de alto funcionamento. \textit{Autism Research}, 10(5), 911-939.

\bibitem{Frontiers2025}
Frontiers in Psychiatry. (2024). Efeitos de diferentes intervenções de exercício na função executiva em crianças com transtorno do espectro autista: uma meta-análise em rede. Recuperado de \url{https://www.frontiersin.org/journals/psychiatry/articles/10.3389/fpsyt.2024.1440123/full}

\bibitem{Frontiers2025Pediatrics}
Frontiers in Pediatrics. (2025). Jogos terapêuticos para autismo: revisão sistemática e meta-análise. \textit{Frontiers in Pediatrics}, 13, Artigo 1801.

\bibitem{Restack2024}
Restack. (2024). IA para estratégias de gamificação para autismo. Recuperado de \url{https://www.restack.io/p/ai-for-gamification-answer-strategies-autism-cat-ai}

\bibitem{Singer1998}
Singer, J. (1998). Pessoas estranhas: O nascimento da comunidade entre pessoas no espectro autista. \textit{University of Technology Sydney}.

\bibitem{WCAG2023}
W3C. (2023). Diretrizes de Acessibilidade para Conteúdo Web (WCAG) 2.1. Recuperado de \url{https://www.w3.org/WAI/WCAG21/quickref/}

\bibitem{Devqube2025}
DevQube. (2025). Neurodiversidade em UX: 7 princípios-chave de design. Recuperado de \url{https://devqube.com/neurodiversity-in-ux/}

\bibitem{Jaeggi2008}
Jaeggi, S. M., Buschkuehl, M., Jonides, J., \& Perrig, W. J. (2008). Melhorando a inteligência fluida com treinamento em memória de trabalho. \textit{Proceedings of the National Academy of Sciences}, 105(19), 6829-6833.

\bibitem{Eriksen1974}
Eriksen, B. A., \& Eriksen, C. W. (1974). Efeitos de letras de ruído na identificação de uma letra-alvo em uma tarefa de não-busca. \textit{Perception \& Psychophysics}, 16(1), 143-149.

\bibitem{Berg1948}
Berg, E. A. (1948). Uma técnica objetiva simples para medir flexibilidade no pensamento. \textit{Journal of General Psychology}, 39(1), 15-22.

\bibitem{Vygotsky1978}
Vygotsky, L. S. (1978). \textit{Mente na sociedade: O desenvolvimento de processos psicológicos superiores}. Harvard University Press.

\bibitem{Hindawi2017}
Hindawi. (2017). Participação em intervenção de habilidades sociais e melhorias associadas no desempenho de função executiva. \textit{Autism Research and Treatment}, 2017, Artigo 5843851.

\end{thebibliography}

\end{document}
